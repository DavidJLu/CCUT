%%%%%%%%%%%%%%%%%%%%%%%%%%%%%%%%%%%%%%%%%
% KOMA-Script Presentation
% LaTeX Template
% Version 1.1 (18/10/15)
%
% This template has been downloaded from:
% http://www.LaTeXTemplates.com
%
% Original Authors:
% Marius Hofert (marius.hofert@math.ethz.ch)
% Markus Kohm (komascript@gmx.info)
% Described in the PracTeX Journal, 2010, No. 2
%
% License:
% CC BY-NC-SA 3.0 (http://creativecommons.org/licenses/by-nc-sa/3.0/)
%
%%%%%%%%%%%%%%%%%%%%%%%%%%%%%%%%%%%%%%%%%

%----------------------------------------------------------------------------------------
%	PACKAGES AND OTHER DOCUMENT CONFIGURATIONS
%----------------------------------------------------------------------------------------

\documentclass[
paper=128mm:96mm, % The same paper size as used in the beamer class
fontsize=11pt, % Font size
pagesize, % Write page size to dvi or pdf
parskip=half-, % Paragraphs separated by half a line
]{scrartcl} % KOMA script (article)

\linespread{1.12} % Increase line spacing for readability

%------------------------------------------------
% Colors
\usepackage{xcolor}	 % Required for custom colors
% Define a few colors for making text stand out within the presentation
\definecolor{mygreen}{RGB}{44,85,17}
\definecolor{myblue}{RGB}{34,31,217}
\definecolor{mybrown}{RGB}{194,164,113}
\definecolor{myred}{RGB}{255,66,56}
% Use these colors within the presentation by enclosing text in the commands below
\newcommand*{\mygreen}[1]{\textcolor{mygreen}{#1}}
\newcommand*{\myblue}[1]{\textcolor{myblue}{#1}}
\newcommand*{\mybrown}[1]{\textcolor{mybrown}{#1}}
\newcommand*{\myred}[1]{\textcolor{myred}{#1}}
%------------------------------------------------

%------------------------------------------------
% Margins
\usepackage[ % Page margins settings
includeheadfoot,
top=3.5mm,
bottom=3.5mm,
left=5.5mm,
right=5.5mm,
headsep=6.5mm,
footskip=8.5mm,
papersize={128mm,96mm}
]{geometry}
%------------------------------------------------

%------------------------------------------------
% Fonts
\usepackage[T1]{fontenc}	 % For correct hyphenation and T1 encoding
\usepackage{lmodern} % Default font: latin modern font
%\usepackage{fourier} % Alternative font: utopia
%\usepackage{charter} % Alternative font: low-resolution roman font
\renewcommand{\familydefault}{\sfdefault} % Sans serif - this may need to be commented to see the alternative fonts
%------------------------------------------------

%------------------------------------------------
% Various required packages
\usepackage{amsthm} % Required for theorem environments
\usepackage{bm} % Required for bold math symbols (used in the footer of the slides)
\usepackage{graphicx} % Required for including images in figures
\usepackage{tikz} % Required for colored boxes
\usepackage{booktabs} % Required for horizontal rules in tables
\usepackage{multicol} % Required for creating multiple columns in slides
\usepackage{lastpage} % For printing the total number of pages at the bottom of each slide
\usepackage[english]{babel} % Document language - required for customizing section titles
\usepackage{microtype} % Better typography
\usepackage{tocstyle} % Required for customizing the table of contents
\usepackage{hyperref}
%------------------------------------------------

%------------------------------------------------
% Slide layout configuration
\usepackage{scrpage2} % Required for customization of the header and footer
\pagestyle{scrheadings} % Activates the pagestyle from scrpage2 for custom headers and footers
\clearscrheadfoot % Remove the default header and footer
\setkomafont{pageheadfoot}{\normalfont\color{black}\sffamily} % Font settings for the header and footer

% Sets vertical centering of slide contents with increased space between paragraphs/lists
\makeatletter
\renewcommand*{\@textbottom}{\vskip \z@ \@plus 1fil}
\newcommand*{\@texttop}{\vskip \z@ \@plus .5fil}
\addtolength{\parskip}{\z@\@plus .25fil}
\makeatother

% Remove page numbers and the dots leading to them from the outline slide
\makeatletter
\newtocstyle[noonewithdot]{nodotnopagenumber}{\settocfeature{pagenumberbox}{\@gobble}}
\makeatother
\usetocstyle{nodotnopagenumber}

\AtBeginDocument{\renewcaptionname{english}{\contentsname}{\Large Outline}} % Change the name of the table of contents
%------------------------------------------------

%------------------------------------------------
% Header configuration - if you don't want a header remove this block
\ihead{
\hspace{-2mm}
\begin{tikzpicture}[remember picture,overlay]
\node [xshift=\paperwidth/2,yshift=-\headheight] (mybar) at (current page.north west)[rectangle,fill,inner sep=0pt,minimum width=\paperwidth,minimum height=2\headheight,top color=mygreen!64,bottom color=mygreen]{}; % Colored bar
\node[below of=mybar,yshift=3.3mm,rectangle,shade,inner sep=0pt,minimum width=128mm,minimum height =1.5mm,top color=black!50,bottom color=white]{}; % Shadow under the colored bar
shadow
\end{tikzpicture}
\color{white}\runninghead} % Header text defined by the \runninghead command below and colored white for contrast
%------------------------------------------------

%------------------------------------------------
% Footer configuration
\setlength{\footheight}{8mm} % Height of the footer
\addtokomafont{pagefoot}{\footnotesize} % Small font size for the footnote

\ifoot{% Left side
\hspace{-2mm}
\begin{tikzpicture}[remember picture,overlay]
\node [xshift=\paperwidth/2,yshift=\footheight] at (current page.south west)[rectangle,fill,inner sep=0pt,minimum width=\paperwidth,minimum height=3pt,top color=mygreen,bottom color=mygreen]{}; % Green bar
\end{tikzpicture}
\myauthor\ \raisebox{0.2mm}{$\bm{\vert}$}\ \myuni % Left side text
}

\ofoot[\pagemark/\pageref{LastPage}\hspace{-2mm}]{\pagemark/\pageref{LastPage}\hspace{-2mm}} % Right side
%------------------------------------------------

%------------------------------------------------
% Section spacing - deeper section titles are given less space due to lesser importance
\usepackage{titlesec} % Required for customizing section spacing
\titlespacing{\section}{0mm}{0mm}{0mm} % Lengths are: left, before, after
\titlespacing{\subsection}{0mm}{0mm}{-1mm} % Lengths are: left, before, after
\titlespacing{\subsubsection}{0mm}{0mm}{-2mm} % Lengths are: left, before, after
\setcounter{secnumdepth}{0} % How deep sections are numbered, set to no numbering by default - change to 1 for numbering sections, 2 for numbering sections and subsections, etc
%------------------------------------------------

%------------------------------------------------
% Theorem style
\newtheoremstyle{mythmstyle} % Defines a new theorem style used in this template
{0.5em} % Space above
{0.5em} % Space below
{} % Body font
{} % Indent amount
{\sffamily\bfseries} % Head font
{} % Punctuation after head
{\newline} % Space after head
{\thmname{#1}\ \thmnote{(#3)}} % Head spec
	
\theoremstyle{mythmstyle} % Change the default style of the theorem to the one defined above
\newtheorem{theorem}{Theorem}[section] % Label for theorems
\newtheorem{remark}[theorem]{Remark} % Label for remarks
\newtheorem{algorithm}[theorem]{Algorithm} % Label for algorithms
\makeatletter % Correct qed adjustment
%------------------------------------------------

%------------------------------------------------
% The code for the box which can be used to highlight an element of a slide (such as a theorem)
\newcommand*{\mybox}[2]{ % The box takes two arguments: width and content
\par\noindent
\begin{tikzpicture}[mynodestyle/.style={rectangle,draw=mygreen,thick,inner sep=2mm,text justified,top color=white,bottom color=white,above}]\node[mynodestyle,at={(0.5*#1+2mm+0.4pt,0)}]{ % Box formatting
\begin{minipage}[t]{#1}
#2
\end{minipage}
};
\end{tikzpicture}
\par\vspace{-1.3em}}
%------------------------------------------------

%----------------------------------------------------------------------------------------
%	PRESENTATION INFORMATION
%----------------------------------------------------------------------------------------

\newcommand*{\mytitle}{First Lecture and Introductions} % Title
\newcommand*{\runninghead}{CS251 at CCUT} % Running head displayed on almost all slides
\newcommand*{\myauthor}{David Lu} % Presenters name(s)
\newcommand*{\mydate}{\today} % Presentation date
\newcommand*{\myuni}{Portland State University --- Department of Computer Science} % University or department

%----------------------------------------------------------------------------------------

\begin{document}

%----------------------------------------------------------------------------------------
%	TITLE SLIDE
%----------------------------------------------------------------------------------------

% Title slide - you may have to tweak a few of the numbers if you wish to make changes to the layout
\thispagestyle{empty} % No slide header and footer
\begin{tikzpicture}[remember picture,overlay] % Background box
\node [xshift=\paperwidth/2,yshift=\paperheight/2] at (current page.south west)[rectangle,fill,inner sep=0pt,minimum width=\paperwidth,minimum height=\paperheight/3,top color=mygreen,bottom color=mygreen]{}; % Change the height of the box, its colors and position on the page here
\end{tikzpicture}
% Text within the box
\begin{flushright}
\vspace{0.6cm}
\color{white}\sffamily
{\bfseries\Large\mytitle\par} % Title
\vspace{0.5cm}
\normalsize
\myauthor\par % Author name
\mydate\par % Date
\vfill
\end{flushright}

\clearpage

%----------------------------------------------------------------------------------------
%	TABLE OF CONTENTS
%----------------------------------------------------------------------------------------

\thispagestyle{empty} % No slide header and footer

\small\tableofcontents % Change the font size and print the table of contents - it may be useful to shrink the font size further if the presentation is full of sections
% To exclude sections/subsections from the table of contents, put an asterisk after \(sub)section like so: \section*{Section Name}

\clearpage

%----------------------------------------------------------------------------------------
%	PRESENTATION SLIDES
%----------------------------------------------------------------------------------------
\section{Hello!}

I am happy to be back to guest teach your class again!

How was your winter?

What have you been learning in the meantime?

How many of your are coming to PSU?

\clearpage

\section{Welcome to CS251!}

Instructor: David Lu \\
Email: dlu@pdx.edu\\
Textbook: \textit{A Concise Introduction to Logic} \\
Author: Craig DeLancey\\
\url{http://textbooks.opensuny.org/concise-introduction-to-logic/} 

Another good textbook: \textit{Forall x}\\
AUthor: P.D. Magnus \\
\url{https://www.fecundity.com/logic/}

Notice both of these are philosophers.

\clearpage

%------------------------------------------------

\subsection{Topics}

CS251 is primarily about one topic: Formal Logic

In this course, we will be studying a number of logical systems, also known
as logical theories or logical systems.  

Logic is important in all areas of study. It's not just for computer science students. Why?



\clearpage

%--------------------------------------------------
\section{Logical Systems}

A logical system consists of four things: 
\begin{enumerate}
	\item A vocabulary of primitive signs used in the language of that system.
	\item A list or set of rules governing what strings of signs (called \textit{formulas})
	are grammatically or syntactically well-formed in the language
	of that system.
	\item A list of axioms, or a subset of the well-formed formulas, considered
	as basic and unprovable principles taken as true in the system.
	\item A specification of what inferences, or inference patterns or rules,
	are taken as valid in that system.
\end{enumerate}


\clearpage
%------------------------------------------------------

\subsection{The Languages}
Because we always start discussing a logical system by discussing the
language it uses, it is worth pausing to discuss the notion of using language
to study language. 

These comprise the first two parts of the logical system: a vocabulary and a syntax or grammar.

\clearpage

%------------------------------------------------

\subsection{Metalanguage and Object Language}

The languages of the systems we study are symbolic logical languages. They use symbols such as $\rightarrow$ and $\lor$, not found in ordinary English or Chinese.

However, we will talk and read \textit{about} these logical languages in ordinary English or Chinese. 

Whenever one language is used to discuss ot study another, we can distinguish between the language that is being studied, called the \textbf{object language}, from the language in which we conduct the study, called the \textbf{metalanguage}.

What one is the object language and which one is the metalanguage for this course?

\clearpage

%------------------------------------------------

\subsection{Object Languages in CS251}

In this course, the object languages will be propositional logic (sometimes called sentential logic) and predicate calculus.

In CS250, set theory was the main object language you studied.

\clearpage

%------------------------------------------------


\subsection{The Logic of the Metalanguage}

Often we will use the metalanguage (English and Chinese) to prove things about the object language.

Proving things already requires logical vocabulary!

Fortunately English (and Chinese) has words like \textit{all}, \textit{or}, \textit{and}, \textit{if}, and so on. These are some of the logical vocabulary of English.

\clearpage

%---------------------------------------------------------------

\section{The Propositional Logic}

For the first part of this class, we will study the Propositional Logic (PL).

\clearpage

\subsection{Logical Vocabulary}

The Propositional Logic, like any  language contains a volcabulary. In this case, it is pretty small, so it is easy to study.

Logical Connectives: $\neg$, $\land$, $\lor$, $\rightarrow$, and $\equiv$ (sometimes $\leftrightarrow$)

Atomic Propositions: Uppercase letters: A, B, C, ... P, Q, R

Sentence Schema (sentence variables): lowercase letters: \textit{p, q, r}

Parentheses: ( ) [ ] \{ \}
\clearpage

\subsection{Syntax}

Any atomic proposition, P, is syntactically well-formed.

For any well-formed proposition, \textit{p}, $\neg p$, is well-formed.

For any well-formed propositions, \textit{p} and \textit{r}, $p \land r$, $p \lor r$, $p \rightarrow r$, and $p \leftrightarrow r$ are well-formed.

\clearpage
%-------------------------------------------------------------------

\section{Precision}

Why do we study these logical languages?

\clearpage

\subsection{Precision}

Answer: We want to use them very precisely.

Consider the precision needed to program a computer. \\
Computers are very dumb. They do exactly what you tell them to.\\
Computer languages are very much like our logical languages -- they are precise.

\clearpage

\subsection{Vagueness}
Natural languages like English and Chinese contain lots of imprecision.

Consider the sentence: \textit{The train is moving too fast.}

Is this true or not? 

\clearpage

\subsection{Ambiguity}

How about this one: \textit{The professor from PSU is very nice.}

This one contains ambiguity and vagueness!

\clearpage
%-----------------------------------------------------------------
\section{Exercises}

Try some exercises from chapter 1 of the textbook

\clearpage

\subsection{Exercise 1}

Vagueness arises when the conditions under which a sentence might be true are not clear. 

Come up with five sentences in English that are vague.

\clearpage

\subsection{Exercise 2}

Ambiguity arises when a word or phrase has several different meanings.

Come up with five sentences in English that are ambiguous.

Hint: This will require that you identify a homonym, two words that sound the same but have different meanings.

\clearpage

\subsection{Exercise 3}

We can often make a vague sentence precise by giving a specific interpretation for the vague term.

For each of the five vague sentences, try to come up with an interpretation that makes the sentence no longer vague.

\clearpage

\subsection{Exercise 4}

Again we can often make ambiguuos sentences precise by specifying which meaning we intended for the ambiguous term.

For each of the five ambiguous sentences, make it precise.

\clearpage

\subsection{Exercise 5}

Come up with five examples of English sentences that are not declarative sentences.
%------------------------------------------------
\clearpage

%------------------------------------------------


%------------------------------------------------

\thispagestyle{empty} % No slide header and footer

\bibliographystyle{unsrt}
\bibliography{sample}

\clearpage

%------------------------------------------------

\thispagestyle{empty} % No slide header and footer

\begin{tikzpicture}[remember picture,overlay] % Background box
\node [xshift=\paperwidth/2,yshift=\paperheight/2] at (current page.south west)[rectangle,fill,inner sep=0pt,minimum width=\paperwidth,minimum height=\paperheight/3,top color=mygreen,bottom color=mygreen]{}; % Change the height of the box, its colors and position on the page here
\end{tikzpicture}
% Text within the box
\begin{flushright}
\vspace{0.6cm}
\color{white}\sffamily
{\bfseries\LARGE Questions?\par} % Request for questions text
\vfill
\end{flushright}

%----------------------------------------------------------------------------------------

\end{document}